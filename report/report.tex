\documentclass[12pt]{article}

\usepackage[utf8]{inputenc}
\usepackage{listings}
\usepackage{graphicx}
\usepackage{float}
\usepackage{geometry}

\usepackage{parskip}
\setlength{\parskip}{1.0\baselineskip plus2pt minus2pt}

\addtolength{\topmargin}{-50pt}
\addtolength{\textheight}{130pt}
\addtolength{\textwidth}{95pt}
\addtolength{\oddsidemargin}{-45pt}

\title{Servidor HTTP - Sistemas Operativos}
\author{David Gomes (2013136061), Miguel Duarte (2012139146)}
\date{Dezembro 2014}

\begin{document}
\maketitle

Neste relatório descrevemos o funcionamento base do nosso
servidor HTTP para Sistemas Operativos.

\section{Scheduler}
Trabalha michel

\section{Configuração}
A nossa configuração é um ficheiro \texttt{.txt} da seguinte forma. A estrutura
da nossa configuração é a seguinte:

\vspace{2mm}
\begin{lstlisting}[language=C]
  struct {
    int n_threads;
    int port;
    int policy_type;

    int n_scripts;
    char scripts[MAX_SCRIPTS][SCRIPT_NAME_STR];
  } typedef config_t;
\end{lstlisting}

\section{Logging e Estatísticas}
Para o \textit{logger} do servidor, usamos filas de mensagens, tal como pedido
no enunciado. A estrutura que enviamos pela fila é a seguinte:

\vspace{2mm}
\begin{lstlisting}[language=C]
  typedef struct {
    long mtype;
    int thread_index;
    char request_type[24];
    char file_name[SIZE_BUF];
  } stats_message;
\end{lstlisting}

\end{document}
