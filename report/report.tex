\documentclass[12pt]{article}

\usepackage[utf8]{inputenc}
\usepackage{listings}
\usepackage{graphicx}
\usepackage{float}
\usepackage{geometry}

\usepackage{parskip}
\setlength{\parskip}{1.0\baselineskip plus2pt minus2pt}

\addtolength{\topmargin}{-50pt}
\addtolength{\textheight}{130pt}
\addtolength{\textwidth}{95pt}
\addtolength{\oddsidemargin}{-45pt}

\title{Servidor HTTP - Sistemas Operativos}
\author{David Gomes (2013136061), Miguel Duarte (2012139146)}
\date{Dezembro 2014}

\begin{document}
\maketitle

Neste relatório descrevemos o funcionamento base do nosso servidor HTTP
para a cadeira de Sistemas Operativos. A compilação e execução do projeto
estão descritas no ficheiro \texttt{README.md} do \textit{source code} do
mesmo.

\section{Scheduler}
O nosso scheduler é uma \textit{thread} responsável por aceder a um \textit{buffer}, criado numa lista ligada, e retirar pedidos segundo uma política de escalonamento definida nas configurações. Caso a política seja FIFO, o primeiro pedido na lista ligada é o primeiro a ser distribuído. Caso se tenha um política de prioridade aos pedidos estáticos, o \textit{buffer} é percorrido até que seja encontrado um pedido deste tipo, que é dado à thread \textit{worker}. Caso seja usado politica de prioridade dinâmica, segue se o mesmo método, mas para encontrar um pedido deste tipo. Nestas políticas, caso não se encontre o tipo de pedido com prioridade, fornece-se o primeiro pedido do \textit{buffer}. Após a selecção do trabalho a processar, vai se encontrar uma \textit{thread} disponível para o fazer.

Na distribuição dos pedidos pelas threads \textit{worker} da \textit{pool}, percorre-se um \textit{array}, com uma posição para cada \textit{thread}, que indica o estado de ocupação de cada. Se for \texttt{1}, está disponível, se for \texttt{0} está ocupada. Naturalmente, cada \textit{thread worker} acede a esse \textit{array} sempre que acaba um trabalho, para se marcar como desocupada, e o \textit{scheduler} marca-a como ocupada sempre que lhe dá serviço. No acesso a esse array, usa-se um pthread\_mutex para cada posição, para garantir sincronização na leitura e escrita. A \textit{thread scheduler} notifica a \textit{thread worker} de que tem trabalho disponível através de uma variável de condição própria para essa \textit{thread}. O trabalho em si é também colocado num \textit{array}, em que cada posição tem o pedido para a \textit{thread} respectiva da \textit{pool}. A \textit{thread scheduler} só acede a este \textit{array} quando a \textit{thread worker} não está em trabalho, naturalmente.

\pagebreak

\section{Tratamento de \textit{requests}}
Os \textit{requests} feitos ao servidor podem ser estáticos ou dinâmicos. Neste
contexto, os pedidos dinâmicos referem-se a \textit{shell scripts} que executamos
no servidor. Para executar os \textit{scripts}, recorremos à função
\texttt{popen}, que cria um processo novo e comunica com este por um \textit{pipe}.
Lemos do \textit{pipe} e o \textit{output} deste é depois devolvido ao utilizador
em formato \texttt{HTML}.

\section{Configuração}
A nossa configuração é um ficheiro de texto que é depois lido numa estrutura
do seguinte tipo:

\vspace{2mm}
\begin{lstlisting}[language=C]
  struct {
    int n_threads;
    int port;
    int policy_type;

    int n_scripts;
    char scripts[MAX_SCRIPTS][SCRIPT_NAME_STR];
  } typedef config_t;
\end{lstlisting}

Apenas permitimos a execução de \textit{scripts} da lista da configuração. Além disto,
temos a configuração a ser relida do ficheiro ao receber um \texttt{SIGHUP}.

\section{Logging e Estatísticas}
Para o \textit{logger} do servidor, usamos filas de mensagens, tal como pedido
no enunciado. A estrutura que enviamos pela fila é a seguinte:

\vspace{2mm}
\begin{lstlisting}[language=C]
  typedef struct {
    long mtype;
    int thread_index;
    char request_type[24];
    char file_name[SIZE_BUF];
  } stats_message;
\end{lstlisting}

\end{document}
